\documentclass[a4paper,11pt]{article}
\input{../macros_docu} % This file is in parent directory. Your TEXINPUTS environment variable must include .. to reach this file. Example: setenv TEXINPUTS ..:../..:${TEXINPUTS}



\graphicspath{{./images/}}
% ---- format de page A4
	\setlength{\textwidth }{16cm}	% largeur de ligne
	\setlength{\textheight}{23cm}   % hauteur du texte
	\setlength{\oddsidemargin}{0cm} % marge pages impaires
	\setlength{\evensidemargin}{0cm}% marge pages paires
	\setlength{\topmargin}{0cm} 	
	\setlength{\headheight}{14pt} 
	\setlength{\headsep}{0.5cm} 

%\newcommand{\trace}{{\mathrm tr}}
\newcommand{\pos}{{\mathbf X}}
\newcommand{\strain}{{\mathbf E}}
\newcommand{\rcgd}{{\mathbf C}}
\newcommand{\spk}{{\mathbf \Sigma}}
\newcommand{\fpk}{{\mathbf P}}
\newcommand{\defGrad}{{\mathbf F}}
\newcommand{\identity}{\mathbf{Id}}
\newcommand{\dir}{\mathbf{e}}
% Title Page
\title{StVenantKirchhoffForceField in Sofa}
%\author{The \sofa{} team}
\date{2014}
\author{Herv\'e Delingette\\ {\small INRIA M\'editerran\'ee, Sophia Antipolis, France}}





\begin{document} 
\maketitle

\section{Patch test}

The density of energy writes as :
\[
 w(\pos)=\frac{\lambda}{2} (\trace \strain)^2 + \mu \trace \strain^2 +\frac{3}{4}(\lambda+2\mu)|J-1|^4_{-}
\]
The second Piola-Kirchhoff $\spk=\frac{\partial w}{\partial \strain}$ then is equal to:
\[
\spk=\lambda (\trace \strain) \identity + \mu \strain +3 J(\lambda+2\mu) |J-1|^3_{-} \rcgd^{-1}
\]
The first  Piola-Kirchhoff $\fpk$ is:
\[
\fpk= \lambda (\trace \strain) \defGrad + \mu \defGrad \strain +3 J(\lambda+2\mu) |J-1|^3_{-} \defGrad^{-T}
\]
Since $\rcgd=\defGrad^T \defGrad$ then $\rcgd=\defGrad^{-1} \defGrad^{-T}$ and $\defGrad \rcgd^{-1}=\defGrad^{-T}$.

In case of uniform extension, we have :
\begin{align*}
\defGrad &=\nabla \Phi=\left [ \begin{array}{ccc} \lambda_1 & 0 & 0 \\0 & \lambda_2 & 0 \\ 0 & 0 & \lambda_3 \end{array} \right ] \\
J &= \lambda_1 \lambda_2 \lambda_3\\
\rcgd &=  \defGrad^T \defGrad = \left [ \begin{array}{ccc} \lambda_1^2 & 0 & 0 \\0 & \lambda_2^2 & 0 \\ 0 & 0 & \lambda_3^2 \end{array} \right ] \\
\strain &= \frac{1}{2}(\rcgd-\identity) = \left [ \begin{array}{ccc} \frac{\lambda_1^2-1}{2} & 0 & 0 \\0 & \frac{\lambda_2^2-1}{2} & 0 \\ 0 & 0 & \frac{\lambda_3^2-1}{2} \end{array} \right ] \\
\spk&=\frac{\lambda (\lambda_1^2+\lambda_2^2+\lambda_3^2-3)}{3} \identity + \mu \strain +3 J(\lambda+2\mu) |J-1|^3_{-}\left [ \begin{array}{ccc} \lambda_1^{-2} & 0 & 0 \\0 & \lambda_2^{-2} & 0 \\ 0 & 0 & \lambda_3^{-2} \end{array} \right ] \\
\fpk&= \defGrad \spk= \frac{\lambda (\lambda_1^2+\lambda_2^2+\lambda_3^2-3)}{3} \defGrad + \mu \defGrad \strain +3 J(\lambda+2\mu) |J-1|^3_{-}\left [ \begin{array}{ccc} \lambda_1^{-1} & 0 & 0 \\0 & \lambda_2^{-1} & 0 \\ 0 & 0 & \lambda_3^{-1} \end{array} \right ] 
\end{align*}

We consider the case of a cube on which a pressure $p$ is applied on its upper face oriented along the $\dir_3$ direction and which is free to deform along the faces oriented along the $\dir_1$ and $\dir_2$ directions. Then we use the 3 relations $\fpk \dir_3=(0 0 p)^T$ and $\fpk \dir_2 =\fpk \dir_1 = {\mathbf 0}$ to obtain a relation between $p$ and $\{\lambda_i\}$. 

More precisely, we obtain :
\begin{align*}
p&=\lambda \lambda_3 \frac{(\lambda_1^2+\lambda_2^2+\lambda_3^2-3)}{3} +\mu \frac{\lambda_3 (\lambda_3^2-1)}{2}+3\lambda_1 \lambda_2  (\lambda+2\mu) |\lambda_1 \lambda_2 \lambda_3-1|^3_{-}\\
0&=\lambda \lambda_1 \frac{(\lambda_1^2+\lambda_2^2+\lambda_3^2-3)}{3} +\mu \frac{\lambda_1 (\lambda_1^2-1)}{2}+3\lambda_2 \lambda_3  (\lambda+2\mu) |\lambda_1 \lambda_2 \lambda_3-1|^3_{-}\\
0&=\lambda \lambda_2 \frac{(\lambda_1^2+\lambda_2^2+\lambda_3^2-3)}{3} +\mu \frac{\lambda_2 (\lambda_2^2-1)}{2}+3\lambda_1 \lambda_3  (\lambda+2\mu) |\lambda_1 \lambda_2 \lambda_3-1|^3_{-}
\end{align*}

If $J>0$ then $\lambda_1=\lambda_2$ according to the last 2 equations. Otherwise, we write $\lambda_2=\alpha \lambda_1$ and then we obtain :
\begin{align*}
 \alpha^2 = \frac{A+B(\lambda_1^2-1)}{ (A+B(\alpha^2\lambda_1^2-1))} \\
 \alpha^2 (A+B(\alpha^2\lambda_1^2-1)) = A+B(\lambda_1^2-1)
\end{align*}
The discriminant is equal to $(A-B)^2+4B\lambda_1(A+B(\lambda_1^2-1))$
\end{document}          

